\section{Evaluation}

\AT{Preamble.}
We pose and answer the following research questions:

\begin{enumerate}
    \item How many successful call signatures are discovered with fuzzing vs. by simply observing calls?
    \AT{This establishes that our technique is useful for expanding the type.}
    \item To what degree does expanding the set of successful call signatures improve coverage?
    \AT{This establishes the usefulness of looking specifically for new call signatures w.r.t. improving coverage, the usual metric for fuzzing papers.}
    \item For those functions where fuzzing does not expand code coverage, do the coverage-guided techniques improve coverage and/or discover more signatures?
    \AT{Hopefully, this shows that our efforts to go beyond basic random value selection are fruitful.}
    \item Are the expanded type signatures useful? \AT{Just a thought; not sure how to measure this. Maybe move to discussion.}
    \item How many bugs are found?
\end{enumerate}

\paragraph{Experiment Server} 
We ran all of our experiments on a \AT{prl3 server specs}.
All reported timing information is \AT{averaged over X runs, with standard deviations reported; timed experiments were conducted on a quiet server with few other processes running to minimize interference.}

%
% RQ1
%

\subsection{How many successful call signatures are discovered with fuzzing vs. by simply observing calls?}

\paragraph{Experimental Design}
\begin{itemize}
    \item We ran \tool on the \TODO{Y} \AT{exported?} functions from \TODO{X} packages.
    \item As a baseline set of successful call signatures, we consider the \textit{origin calls} to the function.
    \item We compare the number \textbf{(?)} of unique successful call signatures reported by \tool with the set of signatures of the origin calls.
\end{itemize}

\paragraph{Results}

\begin{itemize}
    \item \TODO{Table}
\end{itemize}