\section{Approach}

\subsection{Overview}

\subsection{Database of Values}

\begin{itemize}
    \item \AT{We need to discuss the relaxation parameters, since those are central to the fuzzing approach.}
    \item \AT{Also discuss the origin tracking, which is important as input to the fuzzer.}
\end{itemize}

\subsection{Fuzzing Technique}

\begin{itemize}
    \item Say we are fuzzing some function $f$ from a package $p$.
    \item We first consult the database to find all values that were observed as input to $f$; these are our \textit{argument seeds}.
    \item We iteratively generate new calls to $f$ as follows: \ldots first we relax on basically every database parameter \ldots we slowly lower relaxation as the iterations progress (this way, we explore many different parameters at the beginning, and slowly hone-in on values that are likely to work) \ldots
    \item \AT{I'll make this into an algorithm or something.}
\end{itemize}

\subsubsection{Coverage-Guided Fuzzing}

\begin{itemize}
    \item The source code of functions is readily accessible in R.
    \item We parse the code of a function and determine the branch expressions, e.g., the checks in switch and if statements.
    \item We extract those branch expressions \textit{which directly reference function parameters}, and fuzz them separately---if an input causes the expression to evaluate to something we haven't yet seen, we save it.
    \item We can then seed the fuzzer with inputs that are likely to result in different paths through the function, expanding coverage, and expanding the set of valid signatures we observe.
\end{itemize}

\subsubsection{Post-Processing Successful Signatures}

\AT{Discuss everything we do with the successful signatures we get.}

\begin{itemize}
    \item \AT{We can try to consolidate signatures in the same way as the Types for R paper does?}
    \item \AT{We're going to look at metadata related to the return value.}
\end{itemize}