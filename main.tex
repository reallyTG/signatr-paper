\documentclass[sigconf,review,anonymous]{acmart}

\usepackage{authorcomments}


\newcommand{\tool}{\texttt{signatr}\xspace}


%%
%% end of the preamble, start of the body of the document source.
\begin{document}

\title{Fuzzing in Dynamic Languages is Hard}

%%
%% The abstract is a short summary of the work to be presented in the
%% article.
\begin{abstract}
The abstract will go here.
It will be great.
The greatest abstract.

I set up author comments for everyone: \AT{for Alexi}, \PB{for Pierre}, \FK{for Filip}, and \JV{for Jan}, oh and \TODO{for TODOs}.
Feel free to change the colours in {\tt authorcomments.sty}.
\end{abstract}

%%
%% This command processes the author and affiliation and title
%% information and builds the first part of the formatted document.
\maketitle

\section{Introduction}

\subsection{Notes}

We're going to focus on a study of fuzzing for successful inputs in a dynamic language.
Fuzzers need to understand how to call a function, and that's the space we are exploring.
We have a truly ludicrous amount of signatures, and hopefully we can come up with interesting questions and conclusions.

\subsection{Actual Intro Draft}

\begin{itemize}
    \item The permissive semantics of dynamic languages complicate the task of automatically generating inputs for functions, known as \textit{fuzzing}.
    \item In a dynamic language, functions can successfully execute while exhibiting strange behaviour when supplied with unexpected inputs (inputs that would typically be disallowed in more strict, statically typed languages), so it is difficult for fuzzers to gather much feedback about function execution.
    \item Further complicating matters, static analysis of dynamic code is unlikely to yield meaningful insights, and so there is not much information to leverage ahead-of-time to help guide a fuzzer.
    %
    \item \AT{Something about why we want to fuzz dynamic languages.}
    \item \AT{E.g., in R we want to ensure that Notebooks are robust and correct.}
    \item \AT{E.g., server-side JavaScript is becoming increasingly popular, and fuzzing servers for security vulnerabilities would be of great benefit.}
    %
    % \item \AT{Something about types? Need to lead into the question at the centre of the paper.}
    %
    \item One of the first steps of an automated fuzzing tool is to determine what are valid inputs for a function being fuzzed, and our goal in this paper is to answer the question: \textit{how many successful calls can we find?}
    \item \textit{Grammar-based fuzzing} is a technique in which a fuzzer is equipped with a grammar specifying a language of valid inputs \AT{, but defining these grammars can be cumbersome for programmers and they might not be portable.}
    \item \textit{Mutation-based fuzzing} is another technique, in which a fuzzer will mutate a set of previously ``vetted'' inputs (e.g., inputs that were part of a test suite, inputs that explored a new code path through a function, etc.).
    %
    \item In this work, we explore the space of general-purpose fuzzing in dynamic languages, and identify several key challenges. 
    \item We design a fuzzing approach that attempts to find successful calls to a function, with success defined as a call that did not result in an error and did not generate warnings.
    \item Rather than generating inputs, our technique relies on a database of \AT{millions} of unique values we have seen from executing code, and develop a technique to write efficient and expressive queries over this database.
    \item We implement this approach in a fuzzer called \tool for the R programming language, a highly popular and highly dynamic language.
    %
    \item To reiterate, the goal of this work is to shed some light on how difficult it is to automatically build a baseline level of understanding of dynamic code through fuzzing.
    \item To achieve this, we report on two studies: (1) we run \tool on \numPkgsScaleStudy R packages, and report on the number of successful calls that \tool uncovered, and (2) we conduct a case study on \numFnsCaseStudy R functions, wherein we run \tool for an extended period of time and manually analyze the successful calls as well as the code to build an understanding of what constitutes a ``good'' call.
\end{itemize}
\section{Background and Motivation}

\subsection{The R Programming Language}

\subsubsection{Types for R}

\subsection{Fuzzing and Test Generation}

\begin{itemize}
    \item Grammar-based fuzzing:
    \item Mutation-based fuzzing:
\end{itemize}
\section{Approach}

\subsection{Database of Values}

\begin{itemize}
    \item \AT{We need to discuss the relaxation parameters, since those are central to the fuzzing approach.}
    \item \AT{Also discuss the origin tracking, which is important as input to the fuzzer.}
\end{itemize}

\subsection{Fuzzing Technique}

\begin{itemize}
    \item Say we are fuzzing some function $f$ from a package $p$.
    \item We first consult the database to find all values that were observed as input to $f$; these are our \textit{argument seeds}.
    \item We iteratively generate new calls to $f$ as follows: \ldots first we relax on basically every database parameter \ldots we slowly lower relaxation as the iterations progress (this way, we explore many different parameters at the beginning, and slowly hone-in on values that are likely to work) \ldots
    \item \AT{I'll make this into an algorithm or something.}
\end{itemize}

\subsubsection{Coverage-Guided Fuzzing}

\AT{Idea: parse the code, determine branch conditions that involve parameters; run branch conditions on values in the DB to get different results; pass those in in new calls.}
\section{Implementation}

\AT{Implementation details here.}
\section{Evaluation}

\AT{Preamble.}
We pose and answer the following research questions:

\begin{enumerate}
    \item How many successful call signatures are discovered with fuzzing vs. by simply observing calls?
    \AT{This establishes that our technique is useful for expanding the type.}
    \item To what degree does expanding the set of successful call signatures improve coverage?
    \AT{This establishes the usefulness of looking specifically for new call signatures w.r.t. improving coverage, the usual metric for fuzzing papers.}
    \item For those functions where fuzzing does not expand code coverage, do the coverage-guided techniques improve coverage and/or discover more signatures?
    \AT{Hopefully, this shows that our efforts to go beyond basic random value selection are fruitful.}
    \item Are the expanded type signatures useful? \AT{Just a thought; not sure how to measure this. Maybe move to discussion.}
    \item How many bugs are found?
\end{enumerate}

\paragraph{Experiment Server} 
We ran all of our experiments on a \AT{prl3 server specs}.
All reported timing information is \AT{averaged over X runs, with standard deviations reported; timed experiments were conducted on a quiet server with few other processes running to minimize interference.}

%
% RQ1
%

\subsection{How many successful call signatures are discovered with fuzzing vs. by simply observing calls?}

\AT{The idea here is to establish that we uncover more signatures with fuzzing.
We have preliminary results suggesting that this is very likely (the stringr test).}

\paragraph{Experimental Design}
\begin{itemize}
    \item We ran \tool on the \TODO{Y} \AT{exported?} functions from \TODO{X} packages.
    \item Recall the general approach of \tool: for a given package function $f$, we first consult the database for the pre-existing calls to $f$, and take the arguments of those calls as seeds for the fuzzer; then, our test generator iteratively generates new inputs by querying the database for values based on the initial seed.
    \item We compare the signatures generated by this process to the signatures corresponding to the pre-existing calls.
\end{itemize}

\paragraph{Results}

\begin{itemize}
    \item \TODO{Table}
\end{itemize}
\input{sections/case-studies}
\section{Discussion}

\subsection{Expanding the Proposed Type System for R}
\section{Threats to Validity}

\begin{itemize}
    \item \ldots our selection of projects and functions may not be representative \ldots
    \item \AT{others?}
\end{itemize}
\section{Related Work}

\AT{I am using subsections right now to keep the related work grouped, but will probably drop them once there's more stuff here.}

\emph{Randoop}~\cite{pacheco2007randoop} is a feedback-driven random test generation tool for Java, though the technique underpinning it is universally applicable; \AT{in fact, we implemented a version of this technique in R as the baseline for our evaluation}.
The technique described in the \emph{Randoop} paper generates sequences of method calls to test classes, and randomly generates arguments for these calls in two ways: for primitives, a random value is selected from a predefined, but user-extensible list, and for reference types a value is selected at random from those which have been seen, and if none are available then {\tt null} is selected.
This technique is effective at generating tests involving non-trivial objects that are built up from a number of method calls, but these are uncommon in data science languages.

\AT{Add quickcheck}

\subsection{Test Generation for Dynamic Languages}

Many of the aforementioned techniques rely on static function parameter types in creating values with which to call a function, and dynamic languages do not have static type information.
For example, \emph{LambdaTester}~\cite{lambdatester} focuses on test generation for higher-order functions in JavaScript; a discovery phase is required to identify which parameters are expected to be callbacks, and all other, non-callback arguments are generated in a similar manner to \emph{Randoop}.
Further work on \emph{Nessie}~\cite{arteca2022nessie} expanded upon the approach presented in \emph{LambdaTester} to generate tests for asynchronous callbacks using sequencing and nesting.
Other work on fuzzing deep-learning libraries in Python~\cite{wei2022free} explicitly cite Python being a dynamic language as a challenge for test generation; an important part of the pipeline in the paper is inferring types for function parameters by running existing code.

\AT{A few disparate things on Python, Alexi has the links.}

\FK{What relates to our work: (1) retrofitting type systems, (2) mining function specifications (dynamic invariants, (3) evaluating type systems?, (4) type systems for R, (4) tracing.}

%%
%% The next two lines define the bibliography style to be used, and
%% the bibliography file.
\bibliographystyle{ACM-Reference-Format}
\bibliography{fuzzing}


\end{document}
\endinput
%%
%% End of file `sample-sigconf.tex'.
